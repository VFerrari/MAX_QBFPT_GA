\documentclass{MO824}
\usepackage[utf8]{inputenc}
\usepackage[brazil]{babel}      % para texto em Português
\usepackage{amsfonts}
\usepackage{amsmath,amssymb}
\usepackage{graphicx}
\usepackage{xcolor}
\usepackage{hyperref}
\usepackage{multirow}
\usepackage{algorithmic}            
\usepackage[ruled, vlined, portuguese]{algorithm2e}
\newcommand{\Z}{\mathbb{Z}}
\newcommand{\B}{\mathbb{B}}
\newcommand{\N}{\mathbb{N}}
\newcommand{\R}{\mathbb{R}}
\usepackage{changepage}
\usepackage{float}
\usepackage{adjustbox}


\title{Atividade 6 - Grupo 5 - MO824}

\author{
    Rachel Vanucchi Saraiva  - RA 185961\\
    Sinara Caonetto Pamplona - RA 187101\\
    Victor Ferreira Ferrari  - RA 187890
}

\begin{document}

\criartitulo

\runningheads{R.V. Saraiva, S.C. Pamplona e V.F. Ferrari}%
{Atividade 6}

\begin{abstract}
{\bf Resumo}. Este trabalho apresenta soluções heurísticas para o problema MAX-QBFPT baseadas no paradigma de Algoritmos Genéticos. Foram exploradas duas estratégias evolutivas além da padrão:  \textit{Steady-state} e manutenção de diversidade. Os métodos implementados foram avaliados através de 7 experimentos computacionais de 7 instâncias cada. De acordo com os resultados obtidos, as estratégias que proporcionaram resultados melhores foram aquelas que utilizaram \textit{Steady-state} com manutenção de diversidade.

{\bf Palavras-chave}. Algoritmos Genéticos, Otimização, QBFPT, QBF, metaheurística.

\end{abstract}

\section{Introdução}
    \paragraph*{}
    Este trabalho consiste na apresentação de soluções heurísticas, baseadas no paradigma (ou metaheurística) Algoritmos Genéticos, para o problema MAX-QBF com triplas proibidas, proposto na Atividade 6 de MO824 2S-2020. Para isso, foram realizados experimentos testando duas das estratégias evolutivas apresentadas por Colin R. Reeves, em ``Genetic Algorithms'' \cite{ga}.
    
    O Problema MAX-QBF com Triplas Proibidas (MAX-QBFPT) é uma variação do Problema  MAX-QBF (``\textit{Maximum Quadractic Binary Function}''), em que se objetiva maximizar uma função binária quadrática, de modo que $x_i$, $x_j$ e $x_k$ não assumam o valor 1 simultaneamente caso ($i$, $j$, $k$) forme uma tripla proibida. 
    
    \subsection{Modelo Matemático}
    Uma QBF é uma função binária quadrática $f:\B^n \rightarrow \R$ que pode ser expressa como uma soma de termos quadráticos utilizando variáveis binárias. Sejam~$a_{ij} \in \mathbb{R}$ $(i,j = 1,\dots,n)$ os coeficientes de $f$, no problema em que estamos interessados, uma QBF é dada por:
    
    \begin{equation}\label{qbf}
		f(x_1,...,x_n) = \sum_{i=1}^{n} \sum_{j=1}^{n} a_{ij} \cdot x_{i} \cdot x_{j}
	\end{equation}
    
    Sejam $x_i$ as variáveis binárias de uma QBF $f(x_1, \dots, x_n)$, para $i \in \{1 \dots n\}$ e $\mathcal{T} = \{(i,j,k)\in \N : 1 \leq i < j < k \leq n\}$ o conjunto de todas as triplas ordenadas, sem repetição, dos naturais de 1 a $n$. Dado um conjunto de triplas proibidas $T \subseteq \mathcal{T}$, o problema MAX-QBFPT consiste na maximização de $f(x_1, \dots, x_n)$, de modo que $x_i$, $x_j$ e $x_k$ não sejam todos iguais a 1, para cada $(i, j, k) \in T$. Com base nisso, o MAX-QBFPT pode ser formulado como segue:
    
    \begin{minipage}{\linewidth}
		\begin{flalign}\notag
		&& \max Z = \sum_{i=1}^{n} \sum_{j=1}^{n} a_{ij} \cdot x_i \cdot x_j &&\\
		\label{ineq_01}
		&& \text{s.a} \phantom{aaaa} x_i + x_j + x_k \leq  2, && \forall (i, j, k) \in T 			 \\
		\label{ineq_02}
		&& x_i \in \mathbb{B}.          && \forall i \in \{1,...,n\} 
		\end{flalign}~
	\end{minipage}
    
    As restrições \eqref{ineq_01} correspondem às restrições das triplas proibidas, isto é, impedem que toda solução contenha um subconjunto qualquer de triplas ordenadas sem repetições e proibidas. As restrições \eqref{ineq_02} impõem o domínio das variáveis de decisão como sendo binárias. A função objetivo maximiza o valor da QBF.

    
    \subsection{Conjunto de Triplas Proibidas}
    
    Especificamente para esta atividade, as triplas são geradas a partir de duas funções $g,h : [1, n] \rightarrow [1, n]$, que são definidas do seguinte modo:
    
    
    \begin{equation*}
    l_1(u) = 1 + ((\beta_1 \cdot (u - 1) + \beta_2 ) \; mod \; n) 
    \end{equation*}
    
    
    \begin{equation*}
    g(u) = \begin{cases} l_1(u), & \mbox{se } l_1(u) \neq u \\
                        1 + (l_1(u) \; mod \; n), & \mbox{caso contrário } \end{cases}
    \end{equation*}
    
    \begin{equation*}
    l_2(u) = 1 + ((\pi_1 \cdot (u - 1) + \pi_2 ) \; mod \; n) 
    \end{equation*}
    
    
    \begin{equation*}
    h(u) = \begin{cases} l_2(u), & \mbox{se } l_2(u) \neq u  \mbox{ e } l_2(u) \neq g(u)\\
                        1 + (l_2(u) \; mod \; n), & \mbox{se } (1+(l_2(u) \; mod \; n)) \neq u  \mbox{ e } \\ & ~~~~(1+(l_2(u) \; mod \; n)) \neq g(u)\\
                        1 + ((l_2(u)+1) \; mod \; n), & \mbox{caso contrário } \end{cases}
    \end{equation*}

    Onde os valores de $\beta_1$, $\beta_2$, $\pi_1$ e $\pi_2$ são respectivamente 131, 1031, 193 e 1093.
    
    A partir dessas funções definimos o conjunto de triplas proibidas como $T = \{(i,j,k)\in \mathcal{T} : 
    \forall u \in [1,n],  (i,j,k) = sort(\{u, g(u), h(u)\})\}$.

\section{Metodologia} \label{method}

A metodologia a ser utilizada é a de Algoritmo Genético introduzida por John H. Holland em 1975 \cite{johnholland}, metaheurística inspirada pelo conceito biológico de cruzamento seletivo e a forma como os genes dos filhos são combinações dos genes dos pais. Algoritmos Evolutivos, de maneira geral, utilizam conceitos chaves da biologia evolutiva, como hereditariedade, seleção natural, mutação e reprodução/recombinação, sendo estas utilizadas nos experimentos desta atividade.

Como panorama geral da sequência dos passos utilizados em um algoritmo genético, veja abaixo o pseudocódigo do \textit{framework} disponibilizado na disciplina.\\

\begin{algorithm}[H]
		\label{alg_gen}
		\caption{Algoritmo Genético}
		\begin{algorithmic}[1]
		    \STATE populacao = inicializarPopulacao();
		    \STATE melhorCromossomo = melhorCromossomo(populacao);
		    \STATE melhorSolucao = decodificar(melhorCromossomo);\\
		    
		    \FOR {geracao <= maxGeracoes}
		        \STATE {pais = selecionarPais(populacao);}\\
		        \STATE {descendentes = recombinar(pais);}\\
		        \STATE {mutantes = mutacionar(descendentes);}\\
		        \STATE {novaPopulacao = selecionarPopulacao(populacao, mutantes);}\\
		        \STATE {populacao = novaPopulacao;}\\
		        \STATE {melhorCromossomo = melhorCromossomo(populacao);}\\
		        \IF {fitness(melhorCromossomo) > custo(melhorSolucao):}
		            \STATE {melhorSolucao = decodificar(melhorCromossomo);}
		       \ENDIF
		       \IF {tempo-total > tempo-permitido}
		            \STATE {pare;}
		       \ENDIF
		  \ENDFOR
		  
		  \STATE {retorne melhorSolucao;}
		\end{algorithmic}
	\end{algorithm}
	%\vspace{0.5cm}

Em um algoritmo genético, as soluções são codificadas como \textit{cromossomos}, que podem usar diversas estruturas de dados, comumente \textit{strings} binárias. Cada elemento dessa \textit{string} é um \textit{gene}. Para o MAX-QBFPT, cada cromossomo é representado por um vetor de inteiros (são usados apenas 0 e 1), representando \textit{strings} binárias em que o $i$-ésimo gene representa o valor de $x_i$ na solução correspondente. 

A cada iteração é mantido um conjunto de cromossomos, chamado de população, em um vetor. O tamanho da população é parametrizada. A população inicial é gerada de forma aleatória, através da adição de cromossomos retornados por uma função pseudoaleatória que atribui valores binários para cada gene. As próximas populações (gerações) são geradas a partir da recombinação (\textit{crossover}) e mutação da população atual.

Cada cromossomo possui um valor de \textit{fitness}, que serve como medida da qualidade da solução correspondente. Para este problema foi utilizado o próprio valor da solução como \textit{fitness}. Esse valor deve ser maximizado durante a execução, e a melhor solução encontrada é retornada.

Os critérios de parada utilizados foram o número de iterações e o tempo de execução máximo.

\subsection{Viabilização}
Na geração aleatória inicial de soluções, há uma grande possibilidade de violação de triplas proibidas, ou seja, soluções \textbf{inviáveis} para o problema. Inicialmente, foi testado um sistema de penalização de soluções inviáveis, mas em instâncias maiores nenhuma instância viável foi gerada durante a execução inteira.

Por isso, um processo de \textbf{viabilização} foi feito após a geração de cada cromossomo. Cada tripla é guardada em uma lista indexada por seu primeiro elemento. Assim, as triplas localizadas no índice de cada gene ativo são verificadas, em ordem. Caso uma violação seja detectada, um elemento \textbf{aleatório} da tripla é removido da solução, viabilizando o cromossomo em relação àquela tripla e não acarretando na violação de outras.

Uma análise mais complexa pode ser feita para a escolha de remoção, mas realizá-la aleatoriamente ajuda a preservar diversidade e é mais eficiente. O mesmo processo é aplicado nos cromossomos de cada população gerada, entre a mutação e a substituição de gerações, e deste modo a avaliação dos cromossomos é sempre feita com apenas soluções viáveis.

\subsection{Seleção e Reprodução}
Para o problema, foi usado um método \textit{crossover-AND-mutation}, no qual ambas estratégias são usadas em cada geração. O \textit{crossover} é feito para cada par de pais consecutivos em um conjunto previamente selecionado, com o mesmo tamanho da população inicial.

A técnica de \textit{crossover} utilizada no método padrão foi a de \textit{two-point crossover} (2X), na qual duas posições (\textit{crosspoints}) dos cromossomos são escolhidas aleatoriamente. Os genes entre esses dois pontos de cruzamento são trocados entre os cromossomos, gerando dois "filhos", com características de ambos os pais e possivelmente diferentes entre si. 

A escolha dos pais é feita pela estratégia de \textit{seleção por torneio}. Dois cromossomos são escolhidos aleatoriamente e seus valores de \textit{fitness} são comparados. O melhor dos dois cromossomos é escolhido para ser pai. Essa escolha aleatória é feita \textbf{com reposição} até que o conjunto de pais esteja com o tamanho da população, ou seja, com certeza nesse conjunto haverá repetição de pais.

\subsection{Mutação}
No processo de mutação, genes de um cromossomo são escolhidos aleatoriamente e têm seu valor alterado. Como os cromossomos utilizados para o MAX-QBFPT são vetores representando \textit{strings} binárias, essa alteração é simplesmente inverter o valor de um gene. 

A escolha dos genes que sofrem mutação é controlada por um parâmetro, a taxa de mutação, que indica a probabilidade de cada gene ser escolhido.

\subsection{Substituição de Gerações}
No método padrão, a estratégia utilizada para seleção da nova população é a de \textit{elitismo}, na qual a nova população consiste a princípio de todos os filhos gerados pela atual, e o pior cromossomo dos filhos é substituído pelo melhor indivíduo da população atual. Essa estratégia garante que há ao menos uma solução boa da geração anterior preservada.

\subsubsection{Steady-State}
Uma estratégia alternativa também implementada é a chamada \textit{steady-state}, na qual os melhores cromossomos dentre pais e filhos são escolhidos para compor a nova população, de forma a manter o mesmo tamanho da população anterior. Isso é chamado de \textit{steady-state} $\lambda + \mu$ \cite{ga}.

Essa técnica garante que não há perda de soluções boas de uma geração para a próxima, porém pode levar a convergência prematura, e baixa diversidade, graças a uma pressão seletiva alta.

\subsubsection{Manutenção de Diversidade}
Um fator importante para o bom funcionamento de um algoritmo genético é a diversidade da população, para que o algoritmo continue gerando novas soluções e explorando mais o espaço de busca. Como mencionado na seção anterior, o método \textit{steady-state} possui uma pressão seletiva alta, então tende a perder diversidade muito fácil. Por isso, técnicas específicas para manutenção de diversidade são implementadas. 

A técnica implementada neste projeto para a manutenção dessa diversidade consiste na escolha mais cautelosa dos \textit{crosspoints} a serem usados no processo de \textit{crossover}, de forma a evitar a geração de filhos idênticos. Utilizamos a operação XOR entre os pais para identificar os genes que eles diferem entre si como possíveis \textit{crosspoints}. Com essa estratégia, os \textit{crosspoints} são escolhidos aleatoriamente não a partir de todas as posições possíveis de genes, e sim apenas entre aquelas que podem proporcionar maior diversidade no cruzamento. Isso garante que os dois filhos gerados a partir desses pais também terão genes distintos nessa posição. 

\section{Implementação e Avaliação} \label{imp}
    Assim como indicado na atividade, foram utilizados dois tamanhos de população, duas taxas de mutação, e duas estratégias evolutivas alternativas: \textit{steady-state} como método de substituição de população, e manutenção de diversidade para evitar convergência prematura no \textit{steady-state}.
    
    As configurações utilizadas nos experimentos computacionais foram:  
    
\begin{enumerate}
    \item PADRÃO: algoritmo genético com população de tamanho $P_1=100$, taxa de mutação $M_1=0.01$, elitismo sem controle de diversidade.
    
    \item PADRÃO+STSTATE: algoritmo genético PADRÃO mas com \textit{steady-state} como estratégia de substituição de população.
    
    \item PADRÃO+STSTATE+DIV: algoritmo genético PADRÃO mas com \textit{steady-state} como estratégia de substituição de população e manutenção de diversidade.
    
    \item PADRÃO+STSTATE+DIV+POP: algoritmo genético com \textit{steady-state} e manutenção de diversidade, mas com população de tamanho $P_2=50$.
    
    \item PADRÃO+STSTATE+DIV+MUT: algoritmo genético com \textit{steady-state} e manutenção de diversidade, mas com taxa de mutação $M_2=1/n$, onde $n$ é o tamanho da instância.
    
    \item PADRÃO+POP+MUT: algoritmo genético PADRÃO, mas com população de tamanho $P_2=50$ e taxa de mutação $M_2=1/n$, onde $n$ é o tamanho da instância.
    
    \item PADRÃO+STSTATE+DIV+POP+MUT: algoritmo genético com \textit{steady-state} e manutenção de diversidade, mas com população de tamanho $P_2=50$ e taxa de mutação $M_2=1/n$, onde $n$ é o tamanho da instância.
    
\end{enumerate}

    A implementação da solução foi feita em \emph{Java}, com base no \textit{framework} fornecido. Alteramos o código original para incluir o critério de parada por limite de tempo de execução e para implementar as modificações citadas na Seção~\ref{method}.

    As instâncias utilizadas em nossos experimentos foram fornecidas previamente no pacote de atividades. Tais instâncias foram criadas originalmente para o problema MAX-QBF, porém como também foram geradas o conjunto de triplas proibidas, foi possível reaproveitá-las. Algumas instâncias possuem solução ótima conhecida, outras possuem intervalos de referência, que podem ser usados para comparação. A configuração e os limitantes inferiores e superiores do valor ótimo de cada instância são apresentados na Tabela~\ref{tab:instances}.
    
	\begin{table}[h]
	\centering
	\vspace{-0.5cm}
	\caption{Instâncias utilizadas nos experimentos}
    \vspace{0.3cm}
	\label{tab:instances}
	\begin{tabular}{|l|c|c|c|c|}
		\hline
		$i$ & Instância & $n$ & MAX-QBF ($Z^*$) & MAX-QBFPT ($Z^*$) \\ \hline
		1 & qbf020 & 20 & 151 & 125 \\ \hline
		2 & qbf040 & 40 & 429 & 366 \\ \hline
		3 & qbf060 & 60 & 576 & [508, 576] \\ \hline
		4 & qbf080 & 80 & 1000 & [843, 1000] \\ \hline
		5 & qbf100 & 100 & [1468, 1539] & [1263,1539] \\ \hline
		6 & qbf200 & 200 & [5385, 5826] & [3813, 5826] \\ \hline
		7 & qbf400 & 400 & [14826, 16625] & [9645, 16625] \\ \hline
	\end{tabular}
\end{table}
        
    O \textit{hardware} usado para testes possui uma CPU ``Intel(R) Core(TM) i7-8550U (4C/8T)'', com $1.80$ GHz de frequência do \textit{clock}. O computador possui 8 GB de RAM, operando com 2400 MHz. O sistema operacional utilizado foi o Zorin 15.2 (64 bits). O método foi executado com limite de 1800 segundos (30 minutos), limite de 10000 gerações, e sem limite de memória.

\vspace{-0.18cm}
\section{Resultados Obtidos e Análise}
Os resultados obtidos nos testes são mostrados nas Tabelas \ref{tab:conf123}, \ref{tab:conf45} e \ref{tab:conf67}. A iteração na qual o valor máximo foi obtido é apresentada na coluna \textit{Iter}. A corretude da geração de triplas foi testada separadamente, e os testes foram bem-sucedidos.

Os resultados foram de qualidade mista, dados os custos de referência da Tabela \ref{tab:instances}. Para as instâncias maiores, o resultado foi de ótima qualidade, ultrapassando a referência em todas as configurações, muitas vezes por uma grande margem. Nas instâncias para as quais são conhecidas soluções de custo ótimo, elas também foram alcançadas em todas as configurações. Porém, nas instâncias de tamanho médio ($n \in \{60, 80, 100\}$), houve uma grande dificuldade em alcançar os resultados conhecidos. A instância de tamanho 80 em específico foi fonte de dificuldade em conseguir bons resultados, não alcançando a referência em método algum.

Comparando o desempenho das configurações entre si, em termos de qualidade das soluções, a configuração 7 foi em todos os testes melhor ou igual que a configuração padrão, e obteve os melhores resultados para as duas maiores instâncias dentre todas as configurações. As configurações 2 a 6 também obtiveram resultados melhores ou iguais que a configuração 1 para a maior instância, mas obtiveram resultado pior que a configuração padrão para pelo menos uma outra instância. Isso indica que as estratégias alternativas \textit{steady-state} e manutenção de diversidade tiveram um impacto positivo nos resultados, como esperado.

Em relação a tempo de execução, é notável que as soluções baseadas em elitismo chegaram ao número limite de gerações bem mais rapidamente (em alguns casos, mais de 3 vezes mais rapidamente) que as baseadas em \textit{steady-state}. Isso indica que esse método é um dos pontos mais demorados do algoritmo, e para limites de tempo menores, essa técnica pode ser preferível.

A respeito do número de iterações, pode-se ver que em geral (com exceções), as configurações com \textit{steady-state} encontraram suas melhores soluções em um número de gerações bem menor que as correspondentes elitistas, sendo que a maioria dessas soluções são de qualidade similar ou melhor que as elitistas. Assim, para limites menores de gerações, esse método pode ser preferível.

Comparando os resultados dos experimentos tomando em conta a variação dos parâmetros tamanho da população e taxa de mutação, de maneira geral, os experimentos que utilizaram a taxa de mutação $M_2$ apresentaram melhores resultados que as configurações correspondentes com taxa $M_1$ (configuração 5 contra a 3, e configuração 7 contra 4). Isso pode estar relacionado à proporção adequada de mutação de acordo com o tamanho da instância ($1/n$), enquanto que para $M_1$ a taxa é fixa para todas as instâncias. 

Por fim, em relação ao tamanho da população, se compararmos as configurações 1 e 6 que utilizam o mesmo método para selecionar a população (elitismo), percebe-se que a configuração 6 atingiu o limite de gerações mais rapidamente que a configuração 1, pelo fato da população ser menor. Isto também é perceptível entre as configurações 3 e 4. Nesse último par, também é possível ver a queda (em média) na qualidade das soluções encontradas. Isso pode ser explicado pela queda de diversidade com populações menores. A combinação entre menor população e taxa de mutação adaptativa teve bons resultados nas configurações 6 e 7, provavelmente pela maior preservação de diversidade que a maior taxa de mutação fornece. 

\begin{table}[H]
    \centering
     \caption{Tabela com os resultados dos testes com as configurações 1, 2 e 3.}
     \vspace{0.3cm}
      \begin{adjustbox}{max width=\textwidth}
    \begin{tabular}{|c|c|c|c|c|c|c|c|c|c|}
 \hline
 \multicolumn{1}{|c|}{\multirow{2}{*}{$n$}}  & \multicolumn{3}{c|}{Configuração 1}  & \multicolumn{3}{c|}{Configuração 2}  & \multicolumn{3}{c|}{Configuração 3}    \\ \cline{2-10} 
 \multicolumn{1}{|c|}{} & \multicolumn{1}{c|}{Valor} & \multicolumn{1}{c|}{Iter.} & \multicolumn{1}{c|}{Tempo (s)} & \multicolumn{1}{c|}{Valor} & \multicolumn{1}{c|}{Iter.} & \multicolumn{1}{c|}{Tempo (s)} & \multicolumn{1}{c|}{Valor} & \multicolumn{1}{c|}{Iter.} & \multicolumn{1}{c|}{Tempo (s)} \\ \hline
 20 & 125 & 7 & 7,639 & 125 & 5 & 26,402 & 125 & 8 & 40,228\\ \hline
 40 & 366 & 84 & 27,177 & 366 & 19 & 94,592 & 366 & 42 & 145,267\\ \hline
 60 & 495 & 184 & 60,635 & 503 & 41 & 209,324 & 495 & 45 & 319,778\\ \hline
 80 & 809 & 99 & 107,032 & 834 & 66 & 372,714 & 810 & 79 & 583,796\\ \hline
 100 & 1175 & 219 & 157,736 & 1135 & 1475 & 579,611 & 1192 & 64 & 902,505\\ \hline
 200 & 3987 & 3499 & 594,194 & 3831 & 210 & 1800 & 3959 & 285 & 1800\\ \hline
 400 & 10144 & 7160 & 1800 & 11482 & 1276 & 1800 & 11318 & 1122 & 1800\\ \hline
    \end{tabular}
    \end{adjustbox}
    \label{tab:conf123}
\end{table}

\vspace{-0.7cm}

\begin{table}[H]
    \centering
     \caption{Tabela com os resultados dos testes com as configurações 4 e 5.}
     \vspace{0.3cm}
      \begin{adjustbox}{max width=\textwidth}
    \begin{tabular}{|c|c|c|c|c|c|c|}
 \hline
 \multicolumn{1}{|c|}{\multirow{2}{*}{$n$}}  & \multicolumn{3}{c|}{Configuração 4}  & \multicolumn{3}{c|}{Configuração 5}    \\ \cline{2-7} 
 \multicolumn{1}{|c|}{} & \multicolumn{1}{c|}{Valor} & \multicolumn{1}{c|}{Iter.} & \multicolumn{1}{c|}{Tempo (s)} & \multicolumn{1}{c|}{Valor} & \multicolumn{1}{c|}{Iter.} & \multicolumn{1}{c|}{Tempo (s)} \\ \hline
 20 & 125 & 10 & 20,384 & 125 & 2 & 46,818\\ \hline
 40 & 366 & 29 & 71,964 & 366 & 32 & 157,871\\ \hline
 60 & 430 & 6418 & 153,972 & 500 & 49 & 343,318\\ \hline
 80 & 778 & 134 & 274,675 & 810 & 74 & 599,578\\ \hline
 100 & 1189 & 392 & 420,406 & 1192 & 64 & 908,128\\ \hline
 200 & 3893 & 1822 & 1661,908 & 3902 & 227 & 1800\\ \hline
 400 & 11330 & 1306 & 1800 & 11738 & 1041 & 1800\\ \hline
    \end{tabular}
    \end{adjustbox}
    \label{tab:conf45}
\end{table}

\vspace{-0.7cm}

\begin{table}[H]
    \centering
     \caption{Tabela com os resultados dos testes com as configurações 6 e 7.}
     \vspace{0.3cm}
      \begin{adjustbox}{max width=\textwidth}
    \begin{tabular}{|c|c|c|c|c|c|c|}
 \hline
 \multicolumn{1}{|c|}{\multirow{2}{*}{$n$}}  & \multicolumn{3}{c|}{Configuração 6}  & \multicolumn{3}{c|}{Configuração 7}    \\ \cline{2-7} 
 \multicolumn{1}{|c|}{} & \multicolumn{1}{c|}{Valor} & \multicolumn{1}{c|}{Iter.} & \multicolumn{1}{c|}{Tempo (s)} & \multicolumn{1}{c|}{Valor} & \multicolumn{1}{c|}{Iter.} & \multicolumn{1}{c|}{Tempo (s)} \\ \hline
 20 & 125 & 4160 & 6,898 & 125 & 43 & 21,411\\ \hline
 40 & 366 & 46 & 22,071 & 366 & 27 & 73,894\\ \hline
 60 & 508 & 1711 & 46,232 & 495 & 1030 & 153,331\\ \hline
 80 & 826 & 5467 & 82,49 & 813 & 188 & 274,516\\ \hline
 100 & 1189 & 583 & 119,946 & 1189 & 392 & 417,941\\ \hline
 200 & 3983 & 393 & 460,613 & 4038 & 1492 & 1644,922\\ \hline
 400 & 11473 & 2005 & 1800 & 11792 & 1475 & 1800\\ \hline
    \end{tabular}
    \end{adjustbox}
    \label{tab:conf67}
\end{table}

\vspace{-0.75cm}
\bibliographystyle{ieeetr}
{\footnotesize \bibliography{ref}}

\clearpage

\end{document}
